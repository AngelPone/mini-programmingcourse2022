\documentclass[10pt]{beamer}
\usepackage[UTF8]{ctex}
\usepackage{outlines}
\usepackage{hyperref}
\usepackage{minted}
\usepackage{booktabs}
\usemintedstyle{xcode}
\usepackage{tcolorbox}
\tcbuselibrary{minted}

\setminted{
    fontfamily=helvetica
}
\AtBeginSection[]{
  \begin{frame}
  \tableofcontents[currentsection, hideothersubsections]
  \end{frame}
}

\ifx\pdfoutput\undefined
% we are running LaTeX, not pdflatex
\usepackage{graphicx}
\else
% we are running pdflatex, so convert .eps files to .pdf
\usepackage{graphicx}
\usepackage{epstopdf}
\fi

\input{beihangbeamerstyle/beihangcolor}
\input{beihangbeamerstyle/beihangbeamerstyle}

\title{Python入门(四)}
\author{张博涵\\
北京航空航天大学经济管理学院 (\texttt{zhangbohan@buaa.edu.cn})}
\date{\today}


\begin{document}

\begin{frame}
    \maketitle
\end{frame}


\begin{frame}
    \frametitle{Contents}

    \tableofcontents

\end{frame}


\section{函数参数}

\begin{frame}
    \frametitle{函数参数}

    \begin{block}{函数参数}
        \begin{itemize}
            \item 函数可以具有多个参数
            \item 当调用函数时,按照顺序将输入值赋值给函数的局部变量。
        \end{itemize}
    \end{block}

\end{frame}

\begin{frame}
    \frametitle{函数的默认参数与函数调用规则}

    \begin{outline}
        \1 可以为一个或多个参数指定默认值
        \1 在定义函数时,具有默认值的参数必须放在不具有默认值的参数的后面。
        \1 调用函数具有以下三种规则
            \2 所有参数按照顺序调用法,这些参数称为位置参数(positional argument)
            \2 所有参数以\mintinline{python}{arg=value}的形式传入函数,顺序不固定,这些参数称为关键词参数(keyword argument)
            \2 位置与\mintinline{python}{arg=value}混合方式,假设有$n$个参数

            传入的位置参数必须按照顺序,放在前面(即函数的前$k$个局部变量一一对应前$k$个传入的变量)

            后面的$n-k$个以\mintinline{python}{arg=value}调用的参数,不要求顺序
        \1 在调用函数时,没有默认值的参数必须传入,有默认值的参数可选。
        \1 函数定义中的默认值可以是一个变量或表达式,当函数创建时,默认值会在函数定义的空间内求值,并且不再更改,并且在整个存在的范围内指向同一片地址空间,特别是在默认值为可变对象时要尤其注意。一般情况下默认值为不可变对象。
    \end{outline}

\end{frame}

\begin{frame}
    \frametitle{可变参数:*args 以及 **args}

    \begin{block}{在函数定义中使用*args以及**args}
     \begin{itemize}
        \item \mintinline{python}{*args}以及\mintinline{python}{**args}可以作为函数的单独的参数,和普通的命名参数混合使用,\mintinline{python}{*args}必须放在前面。
        \item \mintinline{python}{*args}代表位置参数,\mintinline{python}{**args}代表关键词参数
        \item \mintinline{python}{*args}将传入函数的除一一对应以外的所有位置参数放入一个元组中
        \item \mintinline{python}{**args}将传入函数的除一一对应以外的所有关键词参数放入一个字典中。
        \item 在函数调用时,处于\mintinline{python}{*args}之后的所有参数均必须以关键词参数的形式传入,处于\mintinline{python}{*args}及其之前的参数必须以位置参数的方式传入。
     \end{itemize}
    \end{block}
    

\end{frame}

\begin{frame}
    \frametitle{可变参数:*args 以及 **args}

    \begin{block}{在调用函数时使用*args以及**args}
        \begin{itemize}
            \item 在调用函数时,可以用\mintinline{python}{*args}将一个可迭代对象传入到函数
            \item 也可以用\mintinline{python}{**args}将一个字典传入到函数中
        \end{itemize}
    \end{block}
    

\end{frame}

\begin{frame}[fragile]
    \frametitle{匿名函数:lambda}

    有时候需要临时写一个非常简单的函数,此时去声明一个函数显得复杂冗长,这时候使用匿名函数可以使代码更简洁,速度也可能更快。

    \begin{tcblisting}{listing only, title = Syntax, minted language=python}
lambda arg1,arg2 : expr\end{tcblisting}

\begin{itemize}
    \item 匿名函数没有名字,也没有return,表达式的值即为返回值
    \item 表达式一般只有一句话
    \item 也可以将lambda语句赋值给变量,则该变量称为函数
    \item 经典场景:\mintinline{python}{map}
\end{itemize}
\end{frame}



\section{函数作用域}

\begin{frame}
    \frametitle{函数变量作用域}

    \begin{itemize}
        \item 当变量被传入函数时,开辟了一片新的内存空间用于存储局部变量,对于不可变变量如字符串、数字等,函数调用时将值复制给了局部变量,对于可变变量如列表等将值的内存地址传递给局部变量
        \item 在函数内部对可变变量进行更改,函数外部对应的值也会改变
        \item 在函数中遇到变量时,会首先在当前的函数寻找该变量,若寻找不到,则会在它的父空间中进行寻找,层层寻找一直到找到该变量。
        \item 在函数内部可以访问到函数外部的变量,且可以外部的可变对象进行操作,因此在使用时需要格外注意。
        \item 当函数外部的变量无法访问到函数的局部变量
    \end{itemize}

\end{frame}


\section{模块/module 以及 包/Package}
\begin{frame}[fragile]
    \frametitle{模块/module 以及 包/Package}


    \begin{itemize}
        \item 模块是对一部分具有特定功能的多个Python函数、类、对象、变量等进行组合封装,以组成一个独特的空间,在一些语言中也被称为namespace,一般一个Python文件就是一个模块,模块的名字为文件名。
        \item 包通常包含一个或多个子模块。
        \item 在编程时,通常需要安装和使用别人写好的Package。
    \end{itemize}

    \begin{block}{Package 安装}

        使用\mintinline{python}{pip}安装特定的Package,它会从pypi或其镜像站点上下载安装包并在电脑上安装,镜像站点例如\href{https://mirrors.tuna.tsinghua.edu.cn/help/pypi/}{清华镜像站}
    
    通常可以安装多个Python版本,每个版本都有对应的pip,pip安装的Package会放在该版本的安装路径中。

    \begin{tcblisting}{listing only, minted language=shell, notitle}
        pip install requests
        pip install -i https://pypi.tuna.tsinghua.edu.cn/simple some-package\end{tcblisting}
        
    \end{block}


\end{frame}

\begin{frame}
    \frametitle{导入模块}
    \begin{itemize}
        \item 使用\mintinline{python}{import packagename}导入模块,然后使用\mintinline{python}{packagename.func_name}使用来自模块中的函数或其他对象。
        \item 使用\mintinline{python}{from packagename import object_name},然后直接使用\mintinline{python}{object_name}。
        \item 使用\mintinline{python}{as}进行重命名以避免命名冲突。
        \item \mintinline{python}{from packagename import *}导入所有的对象
        \item 使用\mintinline{python}{import foo.bar}导入子模块
    \end{itemize}

\end{frame}

\begin{frame}
    \frametitle{Python标准库}

    Python提供了丰富的标准库,不需要安装直接导入,包含了丰富的功能,例如
    \begin{itemize}
        \item \mintinline{python}{sys}包含Python本身的基本信息
        \item \mintinline{python}{os}可以与操作系统、文件系统进行交互
        \item \mintinline{python}{math}提供了一些数学计算的函数
        \item \mintinline{python}{time}提供了一些时间相关的操作
    \end{itemize}

\end{frame}

\begin{frame}[fragile]
    \frametitle{作为脚本执行}

    Python的文件既可以作为脚本执行,也可以作为模块被其他文件引入,为了避免在被当做模块引入时引入不必要的变量,可以使用如下方法:

    \begin{tcblisting}{listing only, minted language=python, notitle}
        if __name__ == "__main__":
            # 脚本代码\end{tcblisting}

    \begin{itemize}
        \item \mintinline{python}{__name__}为Python的一个隐藏变量,当作为脚本执行时,为\mintinline{python}{__main__},当作为module被引入时,为文件的名字
    \end{itemize}

\end{frame}

\begin{frame}[fragile]
    \frametitle{作业}

    写一个函数:可以列出当前目录(默认值),或者指定文件目录下的所有后缀为.py的文件,并输出给用户,当目录不存在时输出警告信息例如

    \begin{tcblisting}{listing only, minted language=python, notitle}
>>> find_py()
fibo.py
>>> find_py(".")
fibo.py
>>> find_py("./files")
./files does not exist.\end{tcblisting}

\begin{block}{提示}
    \begin{itemize}
        \item 标准库的os包可以与操作系统进行交互,例如列举出某个文件目录下的文件,判断文件或目录是否存在等
        \item \mintinline{python}{str.endswith(pattern)}方法可以判断字符串后缀是否为\mintinline{python}{pattern}
    \end{itemize}
\end{block}
\end{frame}



\end{document}