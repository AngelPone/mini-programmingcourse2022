\documentclass[9pt]{beamer}
\usepackage[UTF8]{ctex}

\input{bohanbeamerstyle/bohanbeamerstyle}

\title{Python入门(六)}
\author{张博涵}
\institute{北京航空航天大学}
\date{\today}

\begin{document}

\begin{frame}[plain]
\maketitle
\end{frame}


\begin{frame}
    \frametitle{Contents}

    \tableofcontents

\end{frame}

\section{I/O}
\begin{frame}
    \frametitle{I/O Stream}

    在计算机系统中,不同程序之间、程序与系统之间以数据流(Stream)的形式传输数据,也可以称为 I/O Stream,I 代表 Input,O代表Output。

    \vspace{5mm}

    流一般有如下几种途径:

    \begin{outline}
        \1 标准流:标准输入(stdin)、标准输出(stdout)、标准错误(stderr)
        \2 在默认情况下,标准输入流为键盘,标准输出流和错误流为显示器(Shell),Python中通过\mintinline{python}{print}函数打印的内容传入标准输出,\mintinline{python}{input}函数会读取标准输入,报错信息传入标准错误流。
        \2 在Python中,Shell中均可以通过重定向(redirection)将输入流、输出流、错误流定向到文件
        \1 文件流:读取文件内容、向文件写入
    \end{outline}

    \vspace{5mm}

    流具有如下两种类型:

    \begin{itemize}
        \item 文本流:内容是字符串,Python会自动进行编码、解码、转义等,文本流本身有不同的编码方式(字符串章节提过)如UTF-8、GBK、GB2312。
        \item 二进制流:内容是二进制代码,不进行编码解码。
    \end{itemize}
\end{frame}

\begin{frame}[fragile]
    \frametitle{文件读取/写入}

    在Python中进行文件的读取和写入需要首先和文件建立连接,创建一个指向目标的I/O Stream,然后执行读/写操作。

    \begin{block}{读写文件}
        \begin{itemize}
            \item 可以使用\mintinline{python}{open}函数来连接某个文件,主要用法
            
            \begin{codebox}{Python}{open}
open(file_path, mode='r', encoding=None)\end{codebox}

            \item 这里仅列出部分常用参数,其他参数可以查找文档
            \item \mintinline{python}{file_path} 为文件路径
            \item \mintinline{python}{mode}为读写模式,可选包括\mintinline{python}{r, w, a, rb, wb, ab}六种
            \item \mintinline{python}{b}代表二进制流,在读写图像、视频等非文本类数据时常用,\mintinline{Python}{r, w, a}分别代表read(读取), write(写入), append(追加),write和append的区别在于前者会删除文件现有的内容再写入,而后者在文件的末尾写入。
            \item \mintinline{python}{encoding}代表当流为文本流时的编码方式,默认为UTF-8。
        \end{itemize}
    \end{block}
\end{frame}

\begin{frame}[fragile]
    \frametitle{文件读取/写入}
\begin{codebox}{Python}{示例代码}
f = open('test.txt', 'w', encoding='UTF-8')
f.write('你好, test.txt!\n')
f.write('Hello again, test.txt!\n')
f.close()

f = open('test.txt', 'r', encoding='UTF-8')
contents = f.read()
print(contents)
f.close()
\end{codebox}

\begin{itemize}
    \item \mintinline{python}{open}函数返回一个文件对象,可以看做和文件的I/O接口
    \item 在写入模式下,可以通过\mintinline{python}{f.write()}像文件中写入数据
    \item 在读取模式下,可以通过\mintinline{python}{f.read()}读取全部文件内容,也可以通过\mintinline{python}{f.readlines(), f.readline()}等方法分行读取或逐行读取。
    \item 在流使用完毕后,应该借助\mintinline{python}{f.close()}方法关闭该连接,以免错误写入。
\end{itemize}

\end{frame}


\begin{frame}[fragile]
\frametitle{文件读取/写入}
\begin{codebox}{Python}{示例代码}
with open("test.txt", 'r') as f:
    contents = f.read()
print(contents)
\end{codebox}

\begin{itemize}
    \item 为了方便,避免忘记关闭连接,Python提供了\mintinline{python}{with ... as ...} 特殊语法。
    \item 可以在\mintinline{python}{with ... as ...}代码块内执行读取或写入的操作,当离开该代码块时,连接会被自动关闭。
    \item 不止是在文件读写方面,该代码块也在其他地方被使用。
\end{itemize}

\end{frame}


\begin{frame}
    \frametitle{字符串格式化}

    在Python中字符串的格式化方法主要有:

    \begin{itemize}
        \item f-string
        \item \mintinline{python}{'str{},{}'.format()}方法
        \item 其他常用函数
    \end{itemize}

\vspace{5mm}

    参考链接:

    \url{https://docs.python.org/3/tutorial/inputoutput.html}

\end{frame}


\section{错误处理: error handling}


\begin{frame}
    \frametitle{错误: Error}

    由于编程问题导致出现的程序错误,也通常被称为bug,错误主要有如下几种:

    \begin{itemize}
        \item 语法错误 (Syntax error),这种错误最容易发现,因为语法错误时程序无法运行
        \item 逻辑错误 (logic error),这种错误非常常见也是很容易犯的,可能会导致意料之外的结果但不一定会报出错误并导致程序终止,例如在设置条件时小于等于5设置为小于5
        \item 运行时错误 (runtime error),这种错误也非常常见,例如访问只有5个元素的数组的第6个元素,或需要字符串时传入数字,这种错误可以通过测试来避免
    \end{itemize}

    错误处理的主要目的即是在合适的地方报出(raise)错误,并决定处理这种错误,如执行特定的代码,还是终止程序并报出错误信息以提醒用户,主要解决的错误类型为 runtime error。
   
    在Python中,error也常被称为Exception。

\end{frame}

\begin{frame}[fragile]
    \frametitle{Python中的Error}

\begin{codebox}{Python}{示例错误}
>>> 10 * (1/0)
Traceback (most recent call last):
    File "<stdin>", line 1, in <module>
ZeroDivisionError: division by zero
>>> 4 + spam*3
Traceback (most recent call last):
    File "<stdin>", line 1, in <module>
NameError: name 'spam' is not defined
>>> '2' + 2
Traceback (most recent call last):
    File "<stdin>", line 1, in <module>
TypeError: can only concatenate str (not "int") to str
\end{codebox}

\begin{itemize}
    \item \mintinline{python}{traceback}的主要目的是逐层从外到里给出错误信息的报出的地方
    \item \mintinline{python}{NameError, TypeError}等为Error的类型,在Python中,Error主要通过类和对象来实现,也可以通过继承基础的错误类型来实现自己的Error类型。
    \item 冒号后面的内容为错误信息,主要用于提示用户
\end{itemize}

\end{frame}


\begin{frame}[fragile]
    \frametitle{raise error: 报出错误}

    \begin{itemize}
        \item 在Python中最基础的错误类型为\mintinline{python}{BaseException}和\mintinline{python}{Exception},其他的错误类型如\mintinline{Python}{ValueError}, \mintinline{python}{SyntaxError}等继承了这两类实现的。
        \item 报出错误首先需要一个错误类型的对象,然后使用\mintinline{python}{raise}函数报出,在不做任何处理的情况下,会中断程序。
        \item 在初始化对象时,第一个参数为报错信息。
    \end{itemize}

\begin{codebox}{Python}{示例代码}
raise(ValueError("this is a ValueError."))
raise(Exception("this is a normal Exception."))
\end{codebox}
\end{frame}


\begin{frame}[fragile]
\frametitle{处理错误}

可以通过\mintinline{python}{try ... except ...}语法来捕获(capture)并处理错误。

\begin{codebox}{Python}{示例代码}
while True:
    try:
        x = int(input("Please enter a number: "))
        break
    except xxx1:
        # some code
    except (xxx2, xxx3):
        # some code
    finally:
        # some code
    else:
        # some code\end{codebox}
\begin{outline}
    \1 当遇到\mintinline{python}{try}时,会首先执行\mintinline{python}{try}后面紧跟的代码块
    \1 若有错误报出,则根据错误的类型进行匹配并执行对应代码块中的代码,例如若错误类型为\mintinline{python}{ValueError},则执行\mintinline{Python}{print}语句。
    \2 可以有多个\mintinline{python}{except ExceptionType:}分句。
    \2 若匹配不到,则执行\mintinline{python}{finally}语句(可选),否则报出错误,终止程序
    \1 若没有错误,则执行\mintinline{python}{else}后面的代码(如果有的话)。
\end{outline}


\end{frame}





\end{document}