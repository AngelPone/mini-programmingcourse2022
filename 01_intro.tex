\documentclass[10pt]{beamer}
\usepackage[UTF8]{ctex}
\usepackage{outlines}
\usepackage{hyperref}
\ifx\pdfoutput\undefined
% we are running LaTeX, not pdflatex
\usepackage{graphicx}
\else
% we are running pdflatex, so convert .eps files to .pdf
\usepackage{graphicx}
\usepackage{epstopdf}
\fi

\input{beihangbeamerstyle/beihangcolor}
\input{beihangbeamerstyle/beihangbeamerstyle}

\title{计算机世界入门}
\author{张博涵\\
北京航空航天大学经济管理学院 (\texttt{zhangbohan@buaa.edu.cn})}
\date{\today}


\begin{document}
%----------------------------------------------------------------------
% Title frame
\begin{frame}[plain]
\maketitle
\end{frame}

%----------------------------------------------------------------------
% Outline frame
% PLEASE RUN pdflatex TWICE 
\begin{frame}
\frametitle{Outline}
\tableofcontents
\end{frame}
%%=====================================================================
% Section I
\section{引言}
%----------------------------------------------------------------------
% Content frame
\begin{frame}
\frametitle{编程有哪些应用?}
\framesubtitle{大应用}
\begin{block}{操纵硬件}
    操作系统、智能机器人、嵌入式设备、智能家居
\end{block}
\begin{block}{信息的收集、处理、储存、分发}
    互联网、应用程序、数据库、健康码、区块链
\end{block}

\begin{block}{模拟、分析与认知世界}
    大气模拟、导弹弹道模拟、游戏、VR、人工智能
\end{block}
\end{frame}

\begin{frame}
\frametitle{编程有哪些应用?}
\framesubtitle{小应用}
    \begin{outline}
        \1 家庭服务器 
            \2 \url{https://nextcloud.ponez.me:444/}
            \2 \url{https://nextcloud.ponez.me:445/}
            \2 树莓派、智能家居
        \1 爬虫与\href{https://zh.m.wikipedia.org/zh/RSS}{RSS}
            \2 \href{https://docs.rsshub.app/}{RSSHub}
        \1 IFTTT与自动化
            \2 IFTTT: If this then that
            \2 定时任务
    \end{outline}
\end{frame}

%%=====================================================================
% Section II
\section{硬件与操作系统}
%----------------------------------------------------------------------

\begin{frame}
\frametitle{现代计算机的起源}
\framesubtitle{图灵机}
\begin{itemize}
    \item 1936年由英国数学家艾伦·图灵提出,是一种“计算模型”
    \item 可以表达任何有限逻辑数学过程,例如$1+1$
\end{itemize}
\begin{figure}
    \centering
    \includegraphics[width=0.5\textwidth]{figures/tuling.png}    
\end{figure}

\begin{block}{通用图灵机:现代电子计算机的计算模型}
    \begin{itemize}
        \item 将图灵机进行编码$<M>$
        \item 当计算核心遇到编码$<M>$时,模拟图灵机$<M>$的运行
        \item 这些被编码的图灵机称为“程序”
    \end{itemize}
\end{block}
现代计算机采用0-1二进制对数据和程序进行编码

\end{frame}

\begin{frame}
\frametitle{现代计算机的起源}
\framesubtitle{冯·诺伊曼结构}

\begin{itemize}
    \item 冯·诺伊曼结构是实现通用图灵机的一种计算设备
    \item 存储程序计算机
    \item 执行步骤:提取、解码、执行和写回
\end{itemize}

\begin{figure}
    \centering
    \includegraphics[width=0.5\textwidth]{figures/Von_Neumann_architecture..png}
\end{figure}

\end{frame}

\begin{frame}
\frametitle{通用计算设备}
\framesubtitle{以PC(personal computer)为例}

\begin{block}{一般PC的组成}
    \begin{outline}
        \1 CPU:中央处理器,核心的计算单元
        \1 GPU:图形处理器,图形渲染,完全的并行计算
        \1 内存:临时的数据
        \1 硬盘:长期数据存储
        \1 主板:连接各种设备
    \end{outline}
\end{block}

\end{frame}

\begin{frame}
\frametitle{PC与操作系统}
\begin{block}{操作系统}
    \begin{itemize}
        \item 操作系统是人与计算机硬件进行交互的中介。在大多数情况下,只要具有以上的硬件,操作系统可以安装在不同型号的硬件上(“兼容”),但也有些时候它们是绑定的。
        \item PC: Linux、MacOS、Windows
        \item 智能手机:iOS、Android
        \item 其他智能设备
    \end{itemize}
\end{block}

\begin{columns}
    \begin{column}{0.7\textwidth}
        \begin{block}{主要芯片厂商}
            \begin{itemize}
                \item Intel:酷睿系列CPU i3 i5 i7 i9
                \item Apple:M1, M1 pro, M2
                \item AMD:锐龙系列CPU
                \item Nvidia:RTX系列GPU
            \end{itemize}
        \end{block}
    \end{column}
    \begin{column}{0.3\textwidth}
        \begin{block}{CPU架构}
            \begin{itemize}
                
                \item x86
                \item ARM
            \end{itemize}
        \end{block}
    \end{column}
\end{columns}

\end{frame}

\begin{frame}
    \frametitle{PC与操作系统}

        \begin{itemize}
            \item Windows与Linux系统的计算机大多可以自己组装,选择需要的CPU、GPU、内存、硬盘等型号并更换。
            \item MacOS系统由Apple开发,主要应用于自家的Mac电脑产品上,只能通过不正规的渠道安装在自己的硬件上。
        \end{itemize}
\end{frame}

\begin{frame}
\frametitle{Linux与开源}
\begin{figure}
    \centering
    \includegraphics[width=0.1\textwidth]{figures/linux.jpeg}
\end{figure}

\begin{itemize}
    \item 自由和开放源代码的操作系统,社区驱动开发,任何人或组织可以在Linux内核的基础上修改源代码、开发自己的发行版,甚至基于此盈利。
    \item 世界上90\%以上的服务器以及世界500强超级计算机运行在Linux及其变体上,目前在个人电脑上的占有率也逐渐上升。
    \item Windows现在支持Linux内核
    \item 衍生版本:Ubuntu、Debian、CentOS、Deepin等
\end{itemize}
\end{frame}

\section{编程语言}


\begin{frame}
\frametitle{语言类型}

\begin{itemize}
    \item 机器语言:是电脑的CPU 或 GPU 可直接解读的程序,用二进制代码表示
    \item 汇编语言:汇编语言使用助记符(Mnemonics)来代替和表示特定低级机器语言的操作。
    \item 高级语言:以人类的日常语言为基础的编程语言,使用一般人易于接受的文字来表示,有较高的可读性,以方便对电脑认知较浅的人亦可以大概明白其内容,主要为英语。
    \item 现代程序的开发几乎都采用高级语言。
\end{itemize}
\end{frame}

\begin{frame}
\frametitle{现代的高级语言}

\begin{table}
    \centering
    \begin{tabular}{ccccc}
    C &  C++ & Python & Java & R \\
    \includegraphics[width=0.1\textwidth]{figures/C.png} &
    \includegraphics[width=0.1\textwidth]{figures/C++.png} & 
    \includegraphics[width=0.1\textwidth]{figures/python.png} & \includegraphics[width=0.1\textwidth]{figures/java.png} &
    \includegraphics[width=0.1\textwidth]{figures/R.png} \\
    Julia &  Go  & Rust & JavaScript & php \\
    \includegraphics[width=0.1\textwidth]{figures/julia.png} &
    \includegraphics[width=0.1\textwidth]{figures/go.png} & 
    \includegraphics[width=0.1\textwidth]{figures/rust.png} & \includegraphics[width=0.1\textwidth]{figures/javascript.png} &
    \includegraphics[width=0.1\textwidth]{figures/php.png}
    \end{tabular}
\end{table}

除了列出的之外,还有非常非常多。不同的编程语言有各自的特点和擅长的领域,例如C、C++、Rust主要用于系统编程、底层开发等方面;Julia优势在于科学计算方面;R主要用于数据分析与统计建模;Python在人工智能、数据分析、Web开发等方面都应用广泛。

\end{frame}

\section{编程相关软件和环境}

\begin{frame}
\frametitle{Shell}

\begin{itemize}
    \item Shell是操作系统中提供内核访问之服务的工具,是程序员与操作系统进行沟通的桥梁,狭义上来说就是所谓的“命令行界面” + 访问其他应用程序的脚本语言。
    \item 代表性的Shell有Windows下的cmd.exe, PowerShell,Linux及MacOS等类Unix系统下的bash, zsh等。
    \item 在Shell中可以编译和运行程序、操作和处理文件等等
    \item 参考课程:\url{https://missing-semester-cn.github.io/2020/course-shell/} 
\end{itemize}

课后作业一:阅读参考链接并安装Windows Subsystem for Linux


\end{frame}

\begin{frame}
\frametitle{认识文件}

\begin{itemize}
    \item 文件:\textbf{文件名}.\textbf{文件后缀}
    \item 文本文件:只有字符原生编码构成的二进制计算机文件,与富文本相比,其不包含字样样式的控制元素,能够被最简单的文本编辑器直接读取。编程时主要用到的存放源代码的文件。
    \item 富文本文件:docx、pptx, xlsx,需要特定的应用程序进行处理。
    \item 二进制文件:可执行程序,例如微信.exe
    \item 媒体文件:视频、图像等等
\end{itemize}


\end{frame}

\begin{frame}
\frametitle{文本编辑器、IDE}

这里的文本编辑器指用于处理纯文本文件的编辑器。

\begin{itemize}
    \item Windows下的记事本,功能过于简单
    \item Shell:vim、vi、emacs,过于复杂,不适用于新手
    \item 其他专门用于编程的文本编辑器:Visual Studio Code, Sublime Text, Atom等
\end{itemize}

\begin{block}{IDE:集成开发环境}
    将编程的所需要的各个步骤如代码的编辑、编译、测试、发布等集成到一起的辅助程序员的软件工具。
\end{block}

推荐:Visual Studio Code

\end{frame}


\begin{frame}
    \frametitle{练习}
    \begin{itemize}
        \item 安装 Visual Studio Code 以及 Python 并配置Python环境
        \item Hello World!
    \end{itemize}
    
\end{frame}


\begin{frame}
\frametitle{养成好的习惯}

\begin{itemize}
    \item 记笔记
    \item 查资料
    \item Keep your hands dirty!
\end{itemize}
\end{frame}

\begin{frame}
\frametitle{后续课程安排}
\begin{itemize}
    \item Python 入门
    \item 案例:爬虫
    \item 案例:图像压缩
    \item Git、Github与版本控制
    \item ...
\end{itemize}
\end{frame}

\end{document}


