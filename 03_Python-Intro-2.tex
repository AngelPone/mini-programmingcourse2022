\documentclass[10pt]{beamer}
\usepackage[UTF8]{ctex}
\usepackage{outlines}
\usepackage{hyperref}
\usepackage{minted}
\usepackage{booktabs}
\usemintedstyle{xcode}
\setminted{
    fontfamily=helvetica
}
\AtBeginSection[]{
  \begin{frame}
  \tableofcontents[currentsection, hideothersubsections]
  \end{frame}
}
\ifx\pdfoutput\undefined
% we are running LaTeX, not pdflatex
\usepackage{graphicx}
\else
% we are running pdflatex, so convert .eps files to .pdf
\usepackage{graphicx}
\usepackage{epstopdf}
\fi

%--------------------------------------------------------
% NOTE: 1) This is an UNOFFICIAL LaTeX beamer style for 
%           Beihang University.
%       2) This is not exactly a beamer style, rather
%           it contains two LaTeX files to be inserted 
%           in the slides' source file.
%       3) These files are based on Edward Hartley's work
%   <http://www-control.eng.cam.ac.uk/Main/EdwardHartley>
%       4) Complaints or suggestions are always welcome.
%
% Xiaoke Yang (das.xiaoke@hotmail.com)
% Wed 15 Jun 11:02:17 CST 2016
%--------------------------------------------------------


%--------------------------------------------------------
% Set up the Beihang University Colours for use with
% xcolor
%--------------------------------------------------------

% Blue palette
\definecolor{coreBlue}{rgb}{0.02 0.30588 0.64706} %#254aa5


%--------------------------------------------------------
% NOTE: 1) This is an UNOFFICIAL LaTeX beamer style for 
%           Beihang University.
%       2) This is not exactly a beamer style, rather
%           it contains two LaTeX files to be inserted 
%           in the slides' source file.
%       3) These files are based on Edward Hartley's work
%   <http://www-control.eng.cam.ac.uk/Main/EdwardHartley>
%       4) Complaints or suggestions are always welcome.
%
% Xiaoke Yang (das.xiaoke@hotmail.com)
% Wed 15 Jun 11:02:17 CST 2016
%--------------------------------------------------------

%--------------------------------------------------------
% Require tikz to do some text positioning
%--------------------------------------------------------
\usepackage{tikz}

%--------------------------------------------------------
% Use Helvetica rather than Computer Modern Sans Serif
% Comment this out if you prefer Computer Modern
%\usepackage{times}
%--------------------------------------------------------
%\usepackage{helvet}

%--------------------------------------------------------
% If you wish to use Arial, and have the winfonts package
% correctly installed uncomment the following to make the
% default sans serif font Arial
%--------------------------------------------------------
%\usepackage{winfonts}
%\usepackage[T1]{fontenc}
%\renewcommand{\sfdefault}{arial}
%--------------------------------------------------------

%--------------------------------------------------------
% Get rid of the navigation bar 
%--------------------------------------------------------
\beamertemplatenavigationsymbolsempty

%--------------------------------------------------------
% Set the files corresponding to the University crests
% here
%--------------------------------------------------------
% Crest with blue text
\newcommand{\beihangcrestblack}{beihangbeamerstyle/beihang-pantone}

% Crest with white text
\newcommand{\beihangcrestwhite}{beihangbeamerstyle/beihang-rev-pantone}
%--------------------------------------------------------

%--------------------------------------------------------
% Define how the page counter will be displayed on slides
%--------------------------------------------------------
\newcommand{\footlinepagecounter}%
	{\insertframenumber{}/\inserttotalframenumber}
%--------------------------------------------------------

%--------------------------------------------------------
% Set up some lengths
%--------------------------------------------------------
% A paper width for the footline
\newlength{\halfpaperwidth}

% The left margin
\newlength{\headingleftmargin}
% Paper width minus margins
\newlength{\headingwidthminusmargins}
% Height of the heading block
\newlength{\headingheight}
% Height of the footer block
\newlength{\footerheight}

% The height for the titlepageheader in the title page
\newlength{\titlepageheaderheight}
% The height for the footer in the title page
\newlength{\titlepagefooterheight}
% The height for the main title block
\newlength{\titlepagemaintitleblockheight}
% The height for the subtitle block
\newlength{\titlepagesubtitleblockheight}
% The height for the name and date block
\newlength{\titlepagenamedateblockheight}
% The height for the institution block
%\newlength{\titlepageinstitutionheight}

% The lengths for spacing between name and date
\newlength{\titlepagespaceundername}
\newlength{\titlepagespaceunderdate}

% The length for the light blue thin bar


\setlength{\headingleftmargin}{0.05573\paperwidth}
\setlength{\headingwidthminusmargins}{\paperwidth}
\addtolength{\headingwidthminusmargins}{-\headingleftmargin}
\setlength{\headingheight}{0.1459\paperheight}
\setlength{\footerheight}{0.09017\paperheight}

\setlength{\titlepageheaderheight}{0.2361\paperheight}
\setlength{\titlepagefooterheight}{0.1459\paperheight}
\setlength{\titlepagemaintitleblockheight}{0.2361\paperheight}
\setlength{\titlepagesubtitleblockheight}{0.1459\paperheight}
\setlength{\titlepagenamedateblockheight}{0.2361\paperheight}
%\setlength{\titlepageinstitutionheight}{0.95cm}

\setlength{\titlepagespaceundername}{16pt}
\setlength{\titlepagespaceunderdate}{8pt}

%--------------------------------------------------------

%--------------------------------------------------------
% Set up the Beihang University blue scheme for use
% with beamer
%--------------------------------------------------------

% Define colour names
\setbeamercolor{coreBlue}{bg=coreBlue, fg=white}

% Set element colours
\setbeamercolor{subtitle}{fg=white}
\setbeamercolor{titlepageheader}{bg=white,fg=black}
\setbeamercolor{titlepagefooter}{bg=white,fg=black}
\setbeamercolor{block title}{bg=white, fg=coreBlue}
\setbeamercolor{structure}{bg=white, fg=coreBlue}
% \setbeamercolor{alerted text}{fg=darkOrange}


%--------------------------------------------------------
% Set font sizes
%--------------------------------------------------------
\setbeamerfont{frametitle}{size=\large,series=\bfseries}
\setbeamerfont{title}{size=\large,series=\bfseries}
\setbeamerfont{author}{size=\normalsize}
\setbeamerfont{date}{size=\scriptsize}
\setbeamerfont{subtitle}{size=\footnotesize,series=\bfseries}
\setbeamerfont{block title}{size=\normalsize,series=\bfseries}
\setbeamerfont{structure}{size=\normalsize,series=\bfseries}

\setbeamertemplate{itemize item}{\scriptsize\raise1.25pt\hbox{\textbullet}}
\setbeamertemplate{itemize subitem}{\scriptsize\raise1.25pt\hbox{\textbullet}}
\setbeamertemplate{itemize subsubitem}{\scriptsize\raise1.25pt\hbox{\textbullet}}


%-----------------------------------------------------
% Define frame title drawing
%-----------------------------------------------------
\setbeamertemplate{frametitle}
{%
  \nointerlineskip
  \begin{beamercolorbox}[wd=\paperwidth,leftskip=\headingleftmargin]{coreBlue}
    \vskip1pt%
    \tikz{\node[minimum height=\headingheight, inner sep=0cm, text width= \headingwidthminusmargins, text badly ragged]{\usebeamerfont{frametitle}\insertframetitle\\\normalsize\it\insertframesubtitle};}
  \end{beamercolorbox}%
}

%-----------------------------------------------------
% Define footline drawing
%-----------------------------------------------------
\setbeamertemplate{footline}
{%
 \setlength{\halfpaperwidth}{0.5\paperwidth}
 \addtolength{\halfpaperwidth}{1pt}
 \leavevmode
 \begin{beamercolorbox}[sep=0pt,wd=\halfpaperwidth, leftskip=\headingleftmargin,right]{coreBlue}
 \tikz{\node[minimum height=\footerheight, inner sep=0cm]{\footlinepagecounter};}%
 \end{beamercolorbox}
 \hskip-1.5pt%
 \begin{beamercolorbox}[sep=0pt,wd=\halfpaperwidth, leftskip=\headingleftmargin, right,rightskip=\headingleftmargin]{coreBlue}
 \tikz{\node[minimum height=\footerheight, inner sep=0cm]{\includegraphics[width=0.25\paperwidth]{\beihangcrestwhite}};}%
 \end{beamercolorbox}%
}


%-----------------------------------------------------
% Define BEIHANG title page
%-----------------------------------------------------
\setbeamertemplate{title page}
{%
\begin{beamercolorbox}[sep=0cm,right,wd=\paperwidth,ht=\titlepageheaderheight,rightskip=\headingleftmargin]{titlepageheader}
\includegraphics[width=0.3820\paperwidth]{\beihangcrestblack}
\vskip0.2361\titlepageheaderheight
\end{beamercolorbox}
\begin{beamercolorbox}[left,leftskip=\headingleftmargin,wd=\paperwidth,ht=\titlepagemaintitleblockheight]{coreBlue}
\tikz{\node[inner sep=0cm, text width=\paperwidth, minimum height=\titlepagemaintitleblockheight,text badly ragged]{\usebeamerfont{title}\inserttitle};}%
\end{beamercolorbox}%
\nointerlineskip%
\vskip-1pt%
\begin{beamercolorbox}[left,leftskip=\headingleftmargin,wd=\paperwidth,ht=\titlepagesubtitleblockheight]{coreBlue}
\tikz{\node[inner sep=0cm, text width=\paperwidth, minimum height=\titlepagesubtitleblockheight, text badly ragged]{\usebeamerfont{subtitle}\usebeamercolor[fg]{subtitle}\insertsubtitle};}%
\end{beamercolorbox}%
\nointerlineskip%
\vskip-1pt%
\begin{beamercolorbox}[left,leftskip=\headingleftmargin,wd=\paperwidth,ht=\titlepagenamedateblockheight]{coreBlue}
      \usebeamerfont{author}\insertauthor\\
      \vskip\titlepagespaceundername%
      \usebeamerfont{date}\insertdate
      \vskip\titlepagespaceunderdate%
    \end{beamercolorbox}%
\begin{beamercolorbox}[left,leftskip=\headingleftmargin,wd=\paperwidth,ht=\titlepagefooterheight]{titlepagefooter}
\end{beamercolorbox}
}


\title{Python入门(二)}
\author{张博涵\\
北京航空航天大学经济管理学院 (\texttt{zhangbohan@buaa.edu.cn})}
\date{\today}


\begin{document}
\begin{frame}
\maketitle
\end{frame}


\begin{frame}
    \frametitle{Contents}
    \tableofcontents
\end{frame}

\section{复杂数据类型}
\begin{frame}
    \frametitle{列表 - list}

    \begin{block}{关于内存地址}
        \begin{itemize}
            \item 内存地址是一串二进制代码,一般用16进制数来表示: \mintinline{python}{id(var)}
            \item 内存地址表示Python中某个变量指向的数据在内存的位置,CPU通过该地址去访问或修改数据。
        \end{itemize}
    \end{block}

    \begin{block}{列表}
        \begin{itemize}
            \item 列表是数据的最简单的容器形式,可以把多个不同类型的变量或数据放在一起,按顺序进行排列和索引,字符串可以看作为列表的一个特殊形式。
            \item 列表通过\mintinline{python}{[]}进行创建,用整数进行索引,索引从0开始
            \item 列表的内容可替换,而字符串不行。
            \item 列表存储了其中所有元素的内存地址,因此列表的实际内容可能随着元素的变化而变化。
    \end{itemize}
    \end{block}


\end{frame}

\begin{frame}
    \frametitle{列表 - list}

    \begin{table}
    \caption{列表的部分基础操作}    
    \begin{tabular}{ll}
    \toprule
    方法 & 含义 \\ \midrule
    \mintinline{python}{lst.append()} & 在末尾插入 \\
    \mintinline{python}{lst.insert()} & 在指定位置插入 \\
    \mintinline{python}{lst.remove()} & 移除第一次出现的指定值 \\
    \mintinline{python}{len(lst)} & 列表的长度 \\
    \mintinline{python}{lst.reverse()} & 翻转 \\
    \mintinline{python}{lst.pop()} & 移除最后一个 \\
    \mintinline{python}{lst.clear()} & 清空 \\
    \mintinline{python}{lst + lst} & 列表连接 \\
    \mintinline{python}{lst * 2} & 列表重复 \\
    \bottomrule
    \end{tabular}
    \end{table}    

\end{frame}


\begin{frame}
    \frametitle{元组 - tuple}

    元组与列表类似,主要区别如下:
    \begin{itemize}
        \item 用\mintinline{python}{()}创建
        \item 空元组与一个元素的元组:\mintinline{python}{()}, \mintinline{python}{(value,)}
        \item 长度不可变,值不可变
        \item 没有\mintinline{python}{pop}, \mintinline{python}{delete}等方法
        \item 支持\mintinline{python}{+}, \mintinline{python}{*}
    \end{itemize}

\end{frame}

\begin{frame}

    \frametitle{集合 - set}

    \begin{itemize}
        \item 没有重复值的容器,用\mintinline{python}{{}}创建,也可以使用\mintinline{python}{set}函数将可迭代的对象转化成set.
        \item 不支持用整数进行索引
        \item 集合运算
    \end{itemize}

    \begin{table}
        \caption{集合运算}
        \begin{tabular}{ll}
            \toprule
            方法 & 含义 \\
            \midrule
            | & 并集 \\
            \& & 交集 \\
            a \textasciicircum b = (a|b) - (a\&b) & a 和 b 中不同时存在的元素 \\ 
            a - b & 存在于a不存在与b中的元素
        \end{tabular}
    \end{table}

\end{frame}


\begin{frame}
    \frametitle{集合 - set}

    \begin{table}
    \caption{集合常用方法}
    \begin{tabular}{ll} \toprule
    方法 & 含义 \\
    \mintinline{python}{.add} & 增加元素 \\
    \mintinline{python}{.discard} & 删除元素 \\
    \mintinline{python}{.pop} & 随意丢弃一个元素 \\
    \mintinline{python}{.remove} & 丢弃一个指定元素,空时报错 \\
    \mintinline{python}{.issubset} & 是否是另一个集合的子集 \\
    \mintinline{python}{.superset} & 是否是另一个集合的超集 \\ \bottomrule
    \end{tabular}
    \end{table}

\end{frame}

\begin{frame}
    \frametitle{字典 - dict}

    \begin{itemize}
        \item 存储多个键值对,key:value,也可以使用\mintinline{python}{dict()}函数创建
        \item 无序的映射类型,使用key进行索引: \mintinline{python}{mydict[key]}
        \item 使用\mintinline{python}{{key1:value1, key2:value2}}创建
        \item key不可更改,不可重复,值可以更改。
        \item value可以是任意类型,key的类型必须hashable,一般是不可变的对象,例如元组,字符串,数
    \end{itemize}

\end{frame}

\begin{frame}
    \frametitle{字典 - dict}

    \begin{table}
    \begin{tabular}{lcc} \toprule
        方法 & 含义 \\ \midrule
        \mintinline{python}{.keys()} & 列出所有的key \\
        \mintinline{python}{.values()} & 列出所有的value \\
        \mintinline{python}{.items()} & 列出所有的key:value,可迭代 \\
        \mintinline{python}{.pop()} & 删除key以及对应的value \\
        \mintinline{python}{.update()} & 使用另一个字典进行更新 \\ \bottomrule
    \end{tabular}
    \end{table}
\end{frame}


\begin{frame}
    \frametitle{可迭代对象 - Iterable}
    \begin{outline}
        \1 可迭代对象是一种不同数据类型或数据结构遵守的“规范”,主要用于循环。
        \1 主要需要具有这样的功能
            \2 具有一定的顺序
            \2 可以获取第一个,然后获取下一个
        \1 之前讲的list、tuple、set、dict\_keys等可以称为可迭代对象。
        \1 从技术上将,在Python中,所有实现了\mintinline{python}{__iter__()}的方法的对象都是可迭代对象。
    \end{outline}

\end{frame}


\begin{frame}
    \frametitle{类型转换}

    一般来说,可以使用类型的构造器将其他类型转换成对应的类型,例如 \mintinline{python}{int()}, \mintinline{python}{float()}, \mintinline{python}{str()}, \mintinline{python}{dict()}, \mintinline{python}{list()}。

    \begin{itemize}
        \item 将对象转成\mintinline{python}{list()},\mintinline{python}{tuple()}要求输入必须为可迭代对象
        \item 由于部分对象如dict\_keys、set本身是无序的,因此转换成list或进行迭代输出时顺序不确定
    \end{itemize}

\end{frame}

\begin{frame}
    \frametitle{查找文档与帮助}
    不清楚函数的用法时
    \begin{enumerate}
        \item 使用\mintinline{python}{help(func)}
        \item 在交互式环境中使用\mintinline{python}{?}
        \item 在VSCode或Jupyter环境中使用按\mintinline{shell}{tab}可以给出方法提示
        \item 查找Python文档:\href{https://www.python.org/doc/}{官方文档}, \href{https://docs.python.org/3/library/index.html}{Python标准库文档}
        \item bing搜索
    \end{enumerate}

\end{frame}

\end{document}