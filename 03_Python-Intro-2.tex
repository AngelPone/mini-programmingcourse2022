\documentclass[10pt]{beamer}
\usepackage[UTF8]{ctex}
\usepackage{outlines}
\usepackage{hyperref}
\usepackage{minted}
\usepackage{booktabs}
\usemintedstyle{xcode}
\setminted{
    fontfamily=helvetica
}
\AtBeginSection[]{
  \begin{frame}
  \tableofcontents[currentsection, hideothersubsections]
  \end{frame}
}
\ifx\pdfoutput\undefined
% we are running LaTeX, not pdflatex
\usepackage{graphicx}
\else
% we are running pdflatex, so convert .eps files to .pdf
\usepackage{graphicx}
\usepackage{epstopdf}
\fi

\input{beihangbeamerstyle/beihangcolor}
\input{beihangbeamerstyle/beihangbeamerstyle}

\title{Python入门(二)}
\author{张博涵\\
北京航空航天大学经济管理学院 (\texttt{zhangbohan@buaa.edu.cn})}
\date{\today}


\begin{document}
\begin{frame}
\maketitle
\end{frame}


\begin{frame}
    \frametitle{Contents}
    \tableofcontents
\end{frame}

\section{复杂数据类型}
\begin{frame}
    \frametitle{列表 - list}

    \begin{block}{关于内存地址}
        \begin{itemize}
            \item 内存地址是一串二进制代码,一般用16进制数来表示: \mintinline{python}{id(var)}
            \item 内存地址表示Python中某个变量指向的数据在内存的位置,CPU通过该地址去访问或修改数据。
        \end{itemize}
    \end{block}

    \begin{block}{列表}
        \begin{itemize}
            \item 列表是数据的最简单的容器形式,可以把多个不同类型的变量或数据放在一起,按顺序进行排列和索引,字符串可以看作为列表的一个特殊形式。
            \item 列表通过\mintinline{python}{[]}进行创建,用整数进行索引,索引从0开始
            \item 列表的内容可替换,而字符串不行。
            \item 列表存储了其中所有元素的内存地址,因此列表的实际内容可能随着元素的变化而变化。
    \end{itemize}
    \end{block}


\end{frame}

\begin{frame}
    \frametitle{列表 - list}

    \begin{table}
    \caption{列表的部分基础操作}    
    \begin{tabular}{ll}
    \toprule
    方法 & 含义 \\ \midrule
    \mintinline{python}{lst.append()} & 在末尾插入 \\
    \mintinline{python}{lst.insert()} & 在指定位置插入 \\
    \mintinline{python}{lst.remove()} & 移除第一次出现的指定值 \\
    \mintinline{python}{len(lst)} & 列表的长度 \\
    \mintinline{python}{lst.reverse()} & 翻转 \\
    \mintinline{python}{lst.pop()} & 移除最后一个 \\
    \mintinline{python}{lst.clear()} & 清空 \\
    \mintinline{python}{lst + lst} & 列表连接 \\
    \mintinline{python}{lst * 2} & 列表重复 \\
    \bottomrule
    \end{tabular}
    \end{table}    

\end{frame}


\begin{frame}
    \frametitle{元组 - tuple}

    元组与列表类似,主要区别如下:
    \begin{itemize}
        \item 用\mintinline{python}{()}创建
        \item 空元组与一个元素的元组:\mintinline{python}{()}, \mintinline{python}{(value,)}
        \item 长度不可变,值不可变
        \item 没有\mintinline{python}{pop}, \mintinline{python}{delete}等方法
        \item 支持\mintinline{python}{+}, \mintinline{python}{*}
    \end{itemize}

\end{frame}

\begin{frame}

    \frametitle{集合 - set}

    \begin{itemize}
        \item 没有重复值的容器,用\mintinline{python}{{}}创建,也可以使用\mintinline{python}{set}函数将可迭代的对象转化成set.
        \item 不支持用整数进行索引
        \item 集合运算
    \end{itemize}

    \begin{table}
        \caption{集合运算}
        \begin{tabular}{ll}
            \toprule
            方法 & 含义 \\
            \midrule
            | & 并集 \\
            \& & 交集 \\
            a \textasciicircum b = (a|b) - (a\&b) & a 和 b 中不同时存在的元素 \\ 
            a - b & 存在于a不存在与b中的元素
        \end{tabular}
    \end{table}

\end{frame}


\begin{frame}
    \frametitle{集合 - set}

    \begin{table}
    \caption{集合常用方法}
    \begin{tabular}{ll} \toprule
    方法 & 含义 \\
    \mintinline{python}{.add} & 增加元素 \\
    \mintinline{python}{.discard} & 删除元素 \\
    \mintinline{python}{.pop} & 随意丢弃一个元素 \\
    \mintinline{python}{.remove} & 丢弃一个指定元素,空时报错 \\
    \mintinline{python}{.issubset} & 是否是另一个集合的子集 \\
    \mintinline{python}{.superset} & 是否是另一个集合的超集 \\ \bottomrule
    \end{tabular}
    \end{table}

\end{frame}

\begin{frame}
    \frametitle{字典 - dict}

    \begin{itemize}
        \item 存储多个键值对,key:value,也可以使用\mintinline{python}{dict()}函数创建
        \item 无序的映射类型,使用key进行索引: \mintinline{python}{mydict[key]}
        \item 使用\mintinline{python}{{key1:value1, key2:value2}}创建
        \item key不可更改,不可重复,值可以更改。
        \item value可以是任意类型,key的类型必须hashable,一般是不可变的对象,例如元组,字符串,数
    \end{itemize}

\end{frame}

\begin{frame}
    \frametitle{字典 - dict}

    \begin{table}
    \begin{tabular}{lcc} \toprule
        方法 & 含义 \\ \midrule
        \mintinline{python}{.keys()} & 列出所有的key \\
        \mintinline{python}{.values()} & 列出所有的value \\
        \mintinline{python}{.items()} & 列出所有的key:value,可迭代 \\
        \mintinline{python}{.pop()} & 删除key以及对应的value \\
        \mintinline{python}{.update()} & 使用另一个字典进行更新 \\ \bottomrule
    \end{tabular}
    \end{table}
\end{frame}


\begin{frame}
    \frametitle{可迭代对象 - Iterable}
    \begin{outline}
        \1 可迭代对象是一种不同数据类型或数据结构遵守的“规范”,主要用于循环。
        \1 主要需要具有这样的功能
            \2 具有一定的顺序
            \2 可以获取第一个,然后获取下一个
        \1 之前讲的list、tuple、set、dict\_keys等可以称为可迭代对象。
        \1 从技术上将,在Python中,所有实现了\mintinline{python}{__iter__()}的方法的对象都是可迭代对象。
    \end{outline}

\end{frame}


\begin{frame}
    \frametitle{类型转换}

    一般来说,可以使用类型的构造器将其他类型转换成对应的类型,例如 \mintinline{python}{int()}, \mintinline{python}{float()}, \mintinline{python}{str()}, \mintinline{python}{dict()}, \mintinline{python}{list()}。

    \begin{itemize}
        \item 将对象转成\mintinline{python}{list()},\mintinline{python}{tuple()}要求输入必须为可迭代对象
        \item 由于部分对象如dict\_keys、set本身是无序的,因此转换成list或进行迭代输出时顺序不确定
    \end{itemize}

\end{frame}

\begin{frame}
    \frametitle{查找文档与帮助}
    不清楚函数的用法时
    \begin{enumerate}
        \item 使用\mintinline{python}{help(func)}
        \item 在交互式环境中使用\mintinline{python}{?}
        \item 在VSCode或Jupyter环境中使用按\mintinline{shell}{tab}可以给出方法提示
        \item 查找Python文档:\href{https://www.python.org/doc/}{官方文档}, \href{https://docs.python.org/3/library/index.html}{Python标准库文档}
        \item bing搜索
    \end{enumerate}

\end{frame}

\end{document}